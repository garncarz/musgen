\documentclass[a4paper, 12pt]{article}

\usepackage[utf8]{inputenc}
\usepackage[czech]{babel}
\usepackage[IL2]{fontenc}

\usepackage[unicode]{hyperref}
\hypersetup{
	pdftitle={Abstrakt},
	pdfauthor={Ondřej Garncarz}
}

\begin{document}

\author{Ondřej Garncarz}
\title{Generování hudebních skladem počítačem}
\date{}
\maketitle

Práce se zabývá problémem generování hudby počítačem, která je náhodná, ale zároveň ne příliš odlišná od hudby složené člověkem, dodržující určitá pravidla harmonie a nevykračující do disharmonie. Jsou objasněny některé základní pojmy hudební teorie a následně je popsán vyvinutý algoritmus, jednak obecně a~poté i~v~souvislosti k~implementaci v~jazyku Haskell. Program je srovnán s~jemu podobnými, již existujícími, a~je navržen jeho další možný vývoj.

\end{document}

